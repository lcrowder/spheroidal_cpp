\documentclass[12pt]{article}
\usepackage{datetime}
\usepackage{amssymb}
\usepackage{graphicx}
\usepackage{mathptmx}
\usepackage{color}
\usepackage{rotating}
\usepackage[usenames,dvipsnames,table]{xcolor}
\usepackage{amsmath}
\usepackage{bm}
\usepackage{tikz}
\usepackage{pgfplots}

\begin{document}

\title{\vspace{-3cm}{\small{\it Convergence Test Report for}}  \\ \texttt{spheroidal}}

\author{Jacob Spainhour, Leo Crowder}
\date{\currenttime, \today}
\maketitle

\section*{Explanation of Convergence Tests}

Because the evaluation of the double layer potential converges spectrally, we expect that the log of the error should decrease \textbf{linearly} as we increase the order of the spheroidal harmonic expansion used in the method, $p$.
(Note that for an order $p$ expansion, $2p(p+1)$ discretization points are used on the spheroid surface).

To test for this linear relationship, we compute the log10-error over a range of $p$-values, and apply a linear least squares fit.
From running the analogous convergence tests in MATLAB, we expect that for all target points in all three regimes (omitting entries which have reached machine precision, since they obscure the rate of convergence), the slope of the linear fit $\beta_1$ and the residual sum of squares of the fit $RSS$ should satisfy

\begin{gather*}
    -0.45 < \beta_1< -0.2,\\
    RSS < 12.0
\end{gather*}

In each regime we compare our results at 144 target points, a discretization corresponding to $p=8$. 
These points lie on the surface of a spheroid that is scaled up in size from the target geometry by 1.0, 1.1, and 3 for the coincident, near, and far evaluation tests, respectively. 
To examine the convergence, we compute the error of our $D[\sigma]$ computation over a range of orders, $p=2$ to $p=15$. 
We apply these conditions on $\beta_1$ and $RSS$ to each of the three regimes, and print in this document the results of the corresponding Catch2 assertions.

With the exception of this preamble and the introduction describing each of the following tests, this document is generated automatically using data generated by the \texttt{verification\_tests} executable.
\pagebreak

\section*{Coincident Points}
For evaluating $D[\sigma]$ on the coincident spheroid surface, we use as our ``unknown'' dipole density $\sigma(\theta,\phi)=e^{-\cos^2\theta} \sin^2\theta \sin\phi$, selected so that it has a nontrivial expansion in spherical harmonics. 
On this regime, we compare our implementation of evaluating $D[\sigma]$ with reliable singular quadrature.
To ensure high accuracy of singular quadrature, it is evaluated using a discretization corresponding to order $p=16$.

\begin{center}
\def\arraystretch{1.5}
\begin{tabular}{|c|c|c|}
    \hline
    Allowable Slopes & Observed Slopes & Passed Test?\\
    \hline
    \input{data/coincident_assertions.tex}
    \hline
\end{tabular}
\end{center}

\begin{center}
\def\arraystretch{1.5}
\begin{tabular}{|c|c|c|}
    \hline
    RSS Tolerance & Maximum Observed RSS & Passed Test?\\
    \hline
    \input{data/coincident_RSS_assertions.tex}
    \hline
\end{tabular}
\end{center}

\begin{figure}[!ht]
    \centering
    \begin{tikzpicture}
        \begin{axis}[
            xlabel={Order $p$}, ylabel={$\log_{10}( \text{Error} )$},
            ymin=-10, ymax=0, xmin=1, xmax=16,
            legend pos=north east]
          
          \addplot [only marks, mark=*] table {data/coincident_errors.table};
          \addplot [color=red] table {data/coincident_convergence_line.table};
          \legend{, Convergence Line}
        \end{axis}

    \end{tikzpicture}
    \caption{Convergence test for coincident point at index \protect\input{data/coincident_idx.tex}.}
\end{figure}

\pagebreak

\section*{Far Points}
For points that are ``far'' from the spheroid surface, we again take the surface density to be $\sigma(\theta,\phi)=e^{-\cos^2\theta} \sin^2\theta \sin\phi$.
On this spatial regime, the integrand for evaluating the double layer potential is smooth, and we can safely trust smooth quadrature. 
As such, we compare our implementation of $D[\sigma]$ with a Gauss-Legendre plus trapezoidal rule quadrature method of order $p=16$.

\begin{center}
\def\arraystretch{1.5}
\begin{tabular}{|c|c|c|}
    \hline
    Allowable Slopes & Observed Slopes & Passed Test?\\
    \hline
    \input{data/far_assertions.tex}
    \hline
\end{tabular}
\end{center}

\begin{center}
\def\arraystretch{1.5}
\begin{tabular}{|c|c|c|}
    \hline
    RSS Tolerance & Maximum Observed RSS & Passed Test?\\
    \hline
    \input{data/far_RSS_assertions.tex}
    \hline
\end{tabular}
\end{center}

\begin{figure}[!ht]
    \centering
    \begin{tikzpicture}
        \begin{axis}[
            xlabel={Order $p$}, ylabel={$\log_{10}( \text{Error} )$},
            ymin=-10, ymax=0, xmin=1, xmax=16,
            legend pos=north east]
          
          \addplot [only marks, mark=*] table {data/far_errors.table};
          \addplot [color=red] table {data/far_convergence_line.table};
          \legend{, Convergence Line}
        \end{axis}

    \end{tikzpicture}
    \caption{Convergence test for far point at index \protect\input{data/far_idx.tex}.}
\end{figure}

\pagebreak

\section*{Near Points}
Unlike the coincident and far evaluation cases, we cannot use standard quadrature methods to accurately compute the near-singular integrals needed to evaluate $D[\sigma]$ near the spheroid surface.
Instead, we test the evaluation of a density $\sigma$ that corresponds to a simple, known potential $u$, chosen to be the potential due to an arbitrary collection of point charges on the interior of the spheroid.
We then use singular quadrature to construct the matrix operator for $D$ on the surface, and solve the boundary integral equation of an exterior Dirichlet problem for the surface density $\sigma$ that satisfies
$$u(\mathbf{x}) = D[\sigma] + \frac{1}{||\mathbf{x}||}\int \sigma d S$$
for any $\mathbf{x}$ outside of the spheroid. 

Altogether, for our near-surface test, we use our implementation of the double layer routine to compute $D[\sigma]+\frac{1}{||\mathbf{x}||}\int \sigma d S$ and compare with the known true solution $u(\mathbf{x})$ derived from the original point charges.

\begin{center}
\def\arraystretch{1.5}
\begin{tabular}{|c|c|c|}
    \hline
    Allowable Slopes & Observed Slopes & Passed Test?\\
    \hline
    \input{data/near_assertions.tex}
    \hline
\end{tabular}
\end{center}

\begin{center}
\def\arraystretch{1.5}
\begin{tabular}{|c|c|c|}
    \hline
    RSS Tolerance & Maximum Observed RSS & Passed Test?\\
    \hline
    \input{data/near_RSS_assertions.tex}
    \hline
\end{tabular}
\end{center}

\begin{figure}[!ht]
    \centering
    \begin{tikzpicture}
        \begin{axis}[
            xlabel={Order $p$}, ylabel={$\log_{10}( \text{Error} )$},
            ymin=-10, ymax=0, xmin=1, xmax=16,
            legend pos=north east]
          
          \addplot [only marks, mark=*] table {data/near_errors.table};
          \addplot [color=red] table {data/near_convergence_line.table};
          \legend{, Convergence Line}
        \end{axis}

    \end{tikzpicture}
    \caption{Convergence test for near point at index \protect\input{data/near_idx.tex}.}
\end{figure}


\end{document}
